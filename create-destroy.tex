
\section{Creating and Destroying Threads}
\label{sec:createdestroy-threads}

Clearly the first step required in understanding how to build a multi-threaded program is to
understand how to create and destroy threads. There are a number of subtle issues associated
with this topic. Normally one wants to not only create a thread but also to send that thread one
or more parameters. Furthermore when a thread ends, one normally wants to be able to retrieve one
or more values that are returned from the thread. In this section I will describe how these
things can be done with pthreads.

\subsection{Creating Threads}
\label{subsec:creating-threads}

To create a new thread you need to use the \snippet{pthread\_create()} function.
Listing~\ref{lst:skeleton} shows a skeleton program that creates a thread that does nothing and
then waits for the thread to terminate.

\begin{lstlisting}[float=tp,frame=single,xleftmargin=0in,
  caption={Skeleton Thread Program},label=lst:skeleton]  
#include <pthread.h>

/*
 * The function to be executed by the thread should take a
 * void* parameter and return a void* result.
 */
void *thread_function(void *arg)
{
  // Cast the parameter into whatever type is appropriate.
  int *incoming = (int *)arg;

  // Do whatever is necessary using *incoming as the argument.

  // The thread terminates when this function returns.
  return NULL;
}

int main(void)
{
  pthread_t thread_ID;
  void     *thread_result;
  int       value;

  // Put something meaningful into value.
  value = 42;

  // Create the thread, passing &value for the argument.
  pthread_create(&thread_ID, NULL, thread_function, &value);

  // The main program continues while the thread executes.

  // Wait for the thread to terminate.
  pthread_join(thread_ID, &thread_result);

  // Only the main thread is running now.
  return 0;
}
\end{lstlisting}


The \snippet{pthread\_create()} function gives back a thread identifier that can be used in
other calls. The second parameter is a pointer to a \newterm{thread attribute object} that you
can use to set the thread's attributes. The null pointer means to use default attributes which
is suitable for many cases. The third parameter is a pointer to the function the thread is to
execute. The final parameter is the argument passed to the thread function. By using pointers to
void here, any sort of data could potentially be passed provided proper casts are applied. In
the skeleton example I show how a single integer can be used as a thread argument, but in
practice one might send a pointer to a structure containing multiple arguments to the thread.

At some point in your program you should wait for each thread to terminate and collect the
result it produced by calling \snippet{pthread\_join()}. Alternatively you can create a detached
thread. The results returned by such threads are thrown away. The problem with detached threads
is that, unless you make special arrangements, you are never sure when they complete. Usually
you want to make sure all your threads have terminated cleanly before you end the process by
returning from \snippet{main()}.

If you want to kill a thread before its thread function returns normally, you can use
\snippet{pthread\_cancel()}. However, there are difficulties involved in doing that. You must be
sure the thread has released any resources that it has obtained before it actually dies. For
example if a thread has dynamically allocated memory and you cancel it before it can free that
memory, your program will have a memory leak. This is different than when you kill an entire
process. The operating system will typically clean up (certain) resources that are left dangling
by the process. In particular, the entire address space of a process is recovered. However, the
operating system will not do that for a thread since all the threads in a process share
resources. For all the operating system knows, the memory allocated by one thread will be used
by another thread. This situation makes canceling threads carelessly a bad idea.

%
% Write a chapter on thread cancellation and refer to it here?
%

\subsubsection*{Exercises}

\begin{enumerate}

\item Write a program that creates 10 threads. Have each thread execute the same function and
  pass each thread a unique number. Each thread should print ``Hello, World (thread n)'' five
  times where $n$ is replaced by the thread's number. Use an array of \snippet{pthread\_t}
  objects to hold the various thread IDs. Be sure the program doesn't terminate until all the
  threads are complete. Try running your program on more than one machine. Are there any
  differences in how it behaves?

\end{enumerate}

\subsection{Returning Results from Threads}
\label{subsec:returning-results}

The example in the last section illustrated how you can pass an argument into your thread
function if necessary. In this section I will describe how to return results from thread
functions.

Note that the thread functions are declared to return a pointer to \snippet{void}. However,
there are some pitfalls involved in using that pointer. The code below shows one attempt at
returning an integer status code from a thread function.

\begin{lstlisting}
void *thread_function(void *)
{
  int code = DEFAULT_VALUE;

  // Set the value of 'code' as appropriate.

  return (void *)code;
}
\end{lstlisting}

This method will only work on machines where integers can be converted to a pointer and then
back to an integer without loss of information. On some machines such conversions are dangerous.
In fact this method will fail in all cases where one attempts to return an object, such as a
structure, that is larger than a pointer.

In contrast, the code below doesn't fight the type system. It returns a pointer to an internal
buffer where the return value is stored. While the example shows an array of characters for the
buffer, one can easily imagine it being an array of any necessary type, or a single object such
as an integer status code or a structure with many members.

\begin{lstlisting}
void *thread_function(void *)
{
  char buffer[64];

  // Fill up the buffer with something good.

  return buffer;
}
\end{lstlisting}

Alas, the code above fails because the internal buffer is automatic and it vanishes as soon as
the thread function returns. The pointer given back to the calling thread points at undefined
memory. This is another example of the classic dangling pointer error.

In the next attempt the buffer is made static so that it will continue to exist even after the
thread function terminates. This gets around the dangling pointer problem.

\begin{lstlisting}
void *thread_function(void *)
{
  static char buffer[64];

  // Fill up the buffer with something good.

  return buffer;
}
\end{lstlisting}

This method might be satisfactory in some cases, but it doesn't work in the common case of
multiple threads running the same thread function. In such a situation the second thread will
overwrite the static buffer with its own data and destroy that left by the first thread. Global
data suffers from this same problem since global data always has static duration.

The version below is the most general and most robust.

\begin{lstlisting}
void *thread_function(void *)
{
  char *buffer = (char *)malloc(64);

  // Fill up the buffer with something good.

  return buffer;
}
\end{lstlisting}

This version allocates buffer space dynamically. This approach will work correctly even if
multiple threads execute the thread function. Each will allocate a different array and store the
address of that array in a stack variable. Every thread has its own stack so automatic data
objects are different for each thread.

In order to receive the return value of a thread the higher level thread must join with the
subordinate thread. This is shown in the \snippet{main} function of Listing~\ref{lst:skeleton}.
In particular

\begin{lstlisting}
void *thread_result;

// Wait for the thread to terminate.
pthread_join(thread_ID, &thread_result);
\end{lstlisting}

The \snippet{pthread\_join()} function blocks until the thread specified by its first argument
terminates. It then stores into the pointer pointed at by its second argument the value returned
by the thread function. To use this pointer, the higher level thread must cast it into an
appropriate type and dereference it. For example

\begin{lstlisting}
char *message;

message = (char *)thread_result;
printf("I got %s back from the thread.\n", message);
free(thread_result);
\end{lstlisting}

If the thread function allocated the space for the return value dynamically then it is essential
for the higher level thread to free that space when it no longer needs the return value. If this
isn't done the program will leak memory.

\subsubsection*{Exercises}

\begin{enumerate}

\item Write a program that computes the square roots of the integers from 0 to 99 in a separate
  thread and returns an array of doubles containing the results. In the meantime the main thread
  should display a short message to the user and then display the results of the computation
  when they are ready.

\item Imagine that the computations done by the program above were much more time consuming than
  merely calculating a few square roots. Imagine also that displaying the "short message" was
  also fairly time consuming. For example, perhaps the message needed to be fetched from a
  network server as HTML and then rendered. Would you expect the multi-threaded program to
  perform better than a single threaded program that, for example, did the calculations first
  and then fetched the message? Explain.

\end{enumerate}

