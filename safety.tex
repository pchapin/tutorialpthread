
\section{Thread Safety}
\label{sec:thread-safety}

Typically when a complicated object is operated on, it goes through several intermediate,
invalid states before the operation is complete. As an analogy consider what a surgeon does when
he operates on a person. Although the purpose of the operation is to increase the patient's
health, the surgeon performs several steps that would greatly decrease the patient's health if
they were left incomplete! Similarly a function that operates on an object will often temporary
put that object into an unusable state while it performs the update. Once the update is
complete, the function (if written correctly) will leave the object is a fully functional state
again.

Should another thread try to use an object while it is in an unusable state (often called an
\newterm{inconsistent state}) the object will not respond properly and the result will be
undefined. Keeping this from happening is the essential problem of thread safety. The problem
doesn't come up in a single threaded program because there is no possibility of another thread
trying to access the object while the first thread is updating it\footnote{Unless exceptions are
  a possibility. In that case the updating thread might abort the update and then later try to
  access the incompletely updated object. This causes the same sort of problems to occur.}.

\subsection{Levels of Thread Safety}
\label{subsec:levels-safety}

People often have problems discussing thread safety because there are many different levels of
safety one might want to talk about. Just saying that a piece of code is ``thread safe'' doesn't
really say all that much. Yet most people have certain natural expectations about thread safety.
Sometimes those expectations are reasonable and valid, but sometimes they are not. Here are some
of those expectations.

\begin{itemize}
  
\item Reading an object's value with multiple threads is not normally expected to be a problem.
  Problems only occur when an object is updated since it is only then that it has to be modified
  and run the risk of entering inconsistent states.
  
  \emph{However} some objects have internal state that gets modified even when its value is read
  (think about an object that has an internal cache). If two threads try to read such an object
  there might be problems unless the read operations on that object have been designed to handle
  the multiple threads properly.
  
\item Updating two independent objects, even of the same type, is not normally expected to be a
  problem. It is usually assumed that objects that appear to be independent are, in fact,
  independent and thus the inconsistent states of one such object have no impact on the other.
  
  \emph{However} some objects share information behind the scenes (static class data, global
  data, etc) that causes them to be linked internally even when they do not appear to be linked
  from a logical point of view. In that case, modifying two ``independent'' objects might cause
  a problem anyway. Consider:

\begin{lstlisting}
void f()                    void g()
{                           {
  std::string x;              std::string y;

  // Modify x.                // Modify y.
}                           }
\end{lstlisting}
  
If one thread is in function \snippet{f()} modifying the string \snippet{x} and another is in
function \snippet{g()} modifying string \snippet{y}, will there be a problem? Most of the time
you can assume that the two apparently independent objects can be simultaneously modified
without synchronization. But it is possible, depending on how \snippet{std::string} is
implemented, that the two objects share some data internally and that simultaneous modifications
\emph{will} cause a problem. In fact, even if one of the functions merely reads the value of the
string, there might be a problem if they share internal data that is being updated by the other
function.
  
\item Functions that acquire resources, even if from a common pool of resources, are not
  normally expected to be a problem. Consider:

\begin{lstlisting}
void f()                    void g()
{                           {
  char *p = new char[512];    char *p = new char[512];

  // Use the array p.         // Use the array p.
}                           }
\end{lstlisting}
  
If one thread is in function \snippet{f()} and another thread is in function \snippet{g()}, both
threads might try to allocate memory simultaneously by invoking the \snippet{new} operator. In a
multi-threaded environment, it is safe to assume that \snippet{new} has been written to work
correctly in this case even though both invocations of \snippet{new} are trying to take memory
from the same pool of memory. Internally \snippet{new} will synchronize the threads so that each
call will return a unique allocation and the internal memory pool will not be corrupted. Similar
comments can be made about functions that open files, make network connections, and perform
other resource allocation tasks.
  
\emph{However} if the resource allocation functions are not designed with threads in mind then
they may well fail if invoked by multiple threads at once.

\end{itemize}

What people typically expect to cause problems is when a program tries to access (update or
read) an object while another thread is updating that same object. Global objects are
particularly prone to this problem. Local objects are much less so. For example:

\begin{lstlisting}
std::string x;

void f()
{
  std::string y;

  // Modify x and y.
}
\end{lstlisting}

If two threads enter function \snippet{f()} at the same time, they will get different versions
of the string \snippet{y}. This is because every thread has its own stack and local objects are
allocated on the thread's stack. Thus every thread has its own, independent copy of the local
objects. As a result, manipulating \snippet{y} inside \snippet{f()} will not cause a problem
(assuming that manipulating independent objects is safe). However, since there is only one copy
of the global \snippet{x} that both threads will be touching, there could be a problem caused by
those operations.

Local objects are not immune to problems since any function can start a new thread and pass a
pointer to a local object as a parameter to that thread. For example

\begin{lstlisting}
void f()
{
  std::string x;

  start_thread(some_function, &x);
  start_thread(some_function, &x);
}
\end{lstlisting}

Here I assume there is a library function named \snippet{start\_thread()} that accepts a pointer
to a thread function (defined elsewhere) and a pointer to a parameter for that function. In this
case I start two threads executing \snippet{some\_function()}, giving both of them a pointer to
the string \snippet{x}. If \snippet{some\_function()} tries to modify that string then two
threads will be modifying the same object and problems are likely. Note that this case is
particularly insidious because \snippet{some\_function()} has no particular reason to expect
that it will be given the same parameter twice. Thus it is unlikely to have any protection to
handle such a case.

\subsection{Writing Thread Safe Code}
\label{subsec:writing-safety}

In theory the only way to control the actions of a thread is to use synchronization primitives
such as mutexes or semaphores. In languages that provide threads as a part of the language,
synchronization primitives of some kind are normally provided by the language itself. In other
cases, such as with C, they are library functions, such as the POSIX API described in this
tutorial, that interact with the operating system.

Normally you should write code that meets the usual expectations that people have about thread
safe code. If you are implementing a C++ class, make sure that multiple simultaneous reads on an
object are safe. If you do update internal data behind the caller's back, you will probably have
to protect those updates yourself. Also make sure the simultaneous writes to independent objects
are safe. If you do make use of shared data, you will probably have to protect updates to that
shared data yourself. If you write a function that manages shared resources for multiple threads
from a common pool of such resources, you will probably have to protect the resource pool from
corruption by multiple, simultaneous requests. However, in general, you probably don't have to
bother protecting every single object against simultaneous updates. Let the calling program
worry about that case. Such total safety is usually very expensive in terms of runtime
efficiency and not normally necessary or even appropriate.

\subsection{Exception Safety vs Thread Safety}
\label{exception-safety}

Both thread and exception safety share a number of common issues. Both are concerned with
objects that are in an inconsistent state. Both have to think about resources (although in
different ways... exception safety is concerned with resource leaks, thread safety is concerned
with resource corruption). Both have several levels of safety that could be defined along with
some common expectations about what is and is not safe.

However, there is one important difference between the exception safety and thread safety.
Exceptions occur synchronously with the program's execution while threads are asynchronous. In
other words, exceptions occur, in theory, only at certain well defined times. Although it is not
always clear which operations might throw an exception and which might not, in theory it is
possible to precisely define exactly when an exception might happen and when it can't happen. As
a result it is often possible to make a function exception safe by just reorganizing it. In
contrast there is no way to control when two threads might clash. Reorganizing a function is
rarely helpful when it comes to making it thread safe. This difference makes thread related
errors difficult to reproduce and difficult to manage.
