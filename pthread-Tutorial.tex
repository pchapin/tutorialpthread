%%%%%%%%%%%%%%%%%%%%%%%%%%%%%%%%%%%%%%%%%%%%%%%%%%%%%%%%%%%%%%%%%%%%%%%%%%%%
% FILE    : pthread-Tutorial.tex
% AUTHOR  : (C) Copyright 2012 by Peter C. Chapin
% SUBJECT : Tutorial on POSIX threads.
%
% This document attempts to give a quick overview of how to use the facilities of POSIX threads.
% It is not intended to be a complete description of pthreads. Instead I only hope to introduce
% the reader to the basic pthread operations. The hope is that there will be enough information
% here to allow the reader to start using pthreads effectively without overwhelming the reader
% with details.
%
% TODO:
%
% + Add some kind of reference or bibliography section.
%
% + Add some kind of glossary and/or list of abbreviations.
%%%%%%%%%%%%%%%%%%%%%%%%%%%%%%%%%%%%%%%%%%%%%%%%%%%%%%%%%%%%%%%%%%%%%%%%%%%%

%+++++++++++++++++++++++++++++++++
% Preamble and global declarations
%+++++++++++++++++++++++++++++++++
\documentclass{article}
\usepackage{listings}
%\usepackage[dvips]{graphics}
\usepackage{hyperref}
\setlength{\parindent}{0em}
\setlength{\parskip}{1.75ex plus0.5ex minus0.5ex}

% I want to typeset newly introduced terms in a special way.
\newcommand{\newterm}[1]{\emph{#1}}

% Used for bits of C in-line.
\newcommand{\snippet}[1]{\texttt{#1}}

%++++++++++++++++++++
% The document itself
%++++++++++++++++++++
\begin{document}

% The following are settings for the listings package.
\lstset{language=C,
        basicstyle=\small,
        stringstyle=\ttfamily,
        commentstyle=\ttfamily,
        xleftmargin=0.25in,
        showstringspaces=false}

%-----------------------
% Title page information
%-----------------------
\title{pthread Tutorial}
\author{\copyright\ Copyright 2012 by Peter C. Chapin}
\date{September 5, 2012}
\maketitle

\tableofcontents

\section*{Legal}
\label{sec:legal}

\textit{Permission is granted to copy, distribute and/or modify this document under the terms of
  the GNU Free Documentation License, Version 1.1 or any later version published by the Free
  Software Foundation; with no Invariant Sections, with no Front-Cover Texts, and with no
  Back-Cover Texts. A copy of the license is included in the file \texttt{GFDL.txt} distributed
  with the \LaTeX\ source of this document.}

\section{Introduction}
\label{sec:intro}

This document is intended to be a short but useful tutorial on how to use POSIX threads
(pthreads). In this document I do not attempt to give a full description of all pthread
features. Instead I hope to give you enough information to use pthreads in a basic, yet
effective way. Please refer to a text on pthreads for the more esoteric details of the standard.

In addition to talking about the pthread interface itself, I also spend time in this document
discussing issues regarding concurrent programming in general. While such issues are not
specific to pthreads, it is a must that you understand them if you are to use pthreads---or any
thread library---effectively.

I will assume that you are compiling pthreads programs on a Unix system. However, you should be
aware that the pthreads interface is not necessarily specific to Unix. It is a standard
application program interface that could potentially be implemented on many different systems.
However, pthreads is the usual way multi-threaded support is offered in the Unix world. Although
many systems support their own internal method of handling threads, virtually every Unix system
that supports threads at all offers the pthreads interface.

The pthreads API can be implemented either in the kernel of the operating system or in a
library. It can either be preemptive or it can be non-preemptive. A portable program based on
pthreads should not make any assumptions about these matters.

When you compile a program that uses pthreads, you will probably have to set special options on
the compiler's command line to indicate extra (or different) libraries and/or alternative code
generating stratagies. Consult your compiler's documentation for more information on this. Often
you can indicate your desire to use pthreads by supplying the ``-pthread'' option at the end of
the compiler command line. For example

\begin{verbatim}
$ gcc -o myprog myprog.c -pthread
\end{verbatim}

This single option specifies that the pthreads library should be linked and also causes the
compiler to properly handle the multiple threads in the code that it generates.


\section{Creating and Destroying Threads}
\label{sec:createdestroy-threads}

Clearly the first step required in understanding how to build a multi-threaded program is to
understand how to create and destroy threads. There are a number of subtle issues associated
with this topic. Normally one wants to not only create a thread but also to send that thread one
or more parameters. Furthermore when a thread ends, one normally wants to be able to retrieve one
or more values that are returned from the thread. In this section I will describe how these
things can be done with pthreads.

\subsection{Creating Threads}
\label{subsec:creating-threads}

To create a new thread you need to use the \snippet{pthread\_create()} function.
Listing~\ref{lst:skeleton} shows a skeleton program that creates a thread that does nothing and
then waits for the thread to terminate.

\begin{lstlisting}[float=tp,frame=single,xleftmargin=0in,
  caption={Skeleton Thread Program},label=lst:skeleton]  
#include <pthread.h>

/*
 * The function to be executed by the thread should take a
 * void* parameter and return a void* result.
 */
void *thread_function(void *arg)
{
  // Cast the parameter into whatever type is appropriate.
  int *incoming = (int *)arg;

  // Do whatever is necessary using *incoming as the argument.

  // The thread terminates when this function returns.
  return NULL;
}

int main(void)
{
  pthread_t thread_ID;
  void     *thread_result;
  int       value;

  // Put something meaningful into value.
  value = 42;

  // Create the thread, passing &value for the argument.
  pthread_create(&thread_ID, NULL, thread_function, &value);

  // The main program continues while the thread executes.

  // Wait for the thread to terminate.
  pthread_join(thread_ID, &thread_result);

  // Only the main thread is running now.
  return 0;
}
\end{lstlisting}


The \snippet{pthread\_create()} function gives back a thread identifier that can be used in
other calls. The second parameter is a pointer to a \newterm{thread attribute object} that you
can use to set the thread's attributes. The null pointer means to use default attributes which
is suitable for many cases. The third parameter is a pointer to the function the thread is to
execute. The final parameter is the argument passed to the thread function. By using pointers to
void here, any sort of data could potentially be passed provided proper casts are applied. In
the skeleton example I show how a single integer can be used as a thread argument, but in
practice one might send a pointer to a structure containing multiple arguments to the thread.

At some point in your program you should wait for each thread to terminate and collect the
result it produced by calling \snippet{pthread\_join()}. Alternatively you can create a detached
thread. The results returned by such threads are thrown away. The problem with detached threads
is that, unless you make special arrangements, you are never sure when they complete. Usually
you want to make sure all your threads have terminated cleanly before you end the process by
returning from \snippet{main()}.

If you want to kill a thread before its thread function returns normally, you can use
\snippet{pthread\_cancel()}. However, there are difficulties involved in doing that. You must be
sure the thread has released any resources that it has obtained before it actually dies. For
example if a thread has dynamically allocated memory and you cancel it before it can free that
memory, your program will have a memory leak. This is different than when you kill an entire
process. The operating system will typically clean up (certain) resources that are left dangling
by the process. In particular, the entire address space of a process is recovered. However, the
operating system will not do that for a thread since all the threads in a process share
resources. For all the operating system knows, the memory allocated by one thread will be used
by another thread. This situation makes canceling threads carelessly a bad idea.

%
% Write a chapter on thread cancellation and refer to it here?
%

\subsubsection*{Exercises}

\begin{enumerate}

\item Write a program that creates 10 threads. Have each thread execute the same function and
  pass each thread a unique number. Each thread should print ``Hello, World (thread n)'' five
  times where $n$ is replaced by the thread's number. Use an array of \snippet{pthread\_t}
  objects to hold the various thread IDs. Be sure the program doesn't terminate until all the
  threads are complete. Try running your program on more than one machine. Are there any
  differences in how it behaves?

\end{enumerate}

\subsection{Returning Results from Threads}
\label{subsec:returning-results}

The example in the last section illustrated how you can pass an argument into your thread
function if necessary. In this section I will describe how to return results from thread
functions.

Note that the thread functions are declared to return a pointer to \snippet{void}. However,
there are some pitfalls involved in using that pointer. The code below shows one attempt at
returning an integer status code from a thread function.

\begin{lstlisting}
void *thread_function(void *)
{
  int code = DEFAULT_VALUE;

  // Set the value of 'code' as appropriate.

  return (void *)code;
}
\end{lstlisting}

This method will only work on machines where integers can be converted to a pointer and then
back to an integer without loss of information. On some machines such conversions are dangerous.
In fact this method will fail in all cases where one attempts to return an object, such as a
structure, that is larger than a pointer.

In contrast, the code below doesn't fight the type system. It returns a pointer to an internal
buffer where the return value is stored. While the example shows an array of characters for the
buffer, one can easily imagine it being an array of any necessary type, or a single object such
as an integer status code or a structure with many members.

\begin{lstlisting}
void *thread_function(void *)
{
  char buffer[64];

  // Fill up the buffer with something good.

  return buffer;
}
\end{lstlisting}

Alas, the code above fails because the internal buffer is automatic and it vanishes as soon as
the thread function returns. The pointer given back to the calling thread points at undefined
memory. This is another example of the classic dangling pointer error.

In the next attempt the buffer is made static so that it will continue to exist even after the
thread function terminates. This gets around the dangling pointer problem.

\begin{lstlisting}
void *thread_function(void *)
{
  static char buffer[64];

  // Fill up the buffer with something good.

  return buffer;
}
\end{lstlisting}

This method might be satisfactory in some cases, but it doesn't work in the common case of
multiple threads running the same thread function. In such a situation the second thread will
overwrite the static buffer with its own data and destroy that left by the first thread. Global
data suffers from this same problem since global data always has static duration.

The version below is the most general and most robust.

\begin{lstlisting}
void *thread_function(void *)
{
  char *buffer = (char *)malloc(64);

  // Fill up the buffer with something good.

  return buffer;
}
\end{lstlisting}

This version allocates buffer space dynamically. This approach will work correctly even if
multiple threads execute the thread function. Each will allocate a different array and store the
address of that array in a stack variable. Every thread has its own stack so automatic data
objects are different for each thread.

In order to receive the return value of a thread the higher level thread must join with the
subordinate thread. This is shown in the \snippet{main} function of Listing~\ref{lst:skeleton}.
In particular

\begin{lstlisting}
void *thread_result;

// Wait for the thread to terminate.
pthread_join(thread_ID, &thread_result);
\end{lstlisting}

The \snippet{pthread\_join()} function blocks until the thread specified by its first argument
terminates. It then stores into the pointer pointed at by its second argument the value returned
by the thread function. To use this pointer, the higher level thread must cast it into an
appropriate type and dereference it. For example

\begin{lstlisting}
char *message;

message = (char *)thread_result;
printf("I got %s back from the thread.\n", message);
free(thread_result);
\end{lstlisting}

If the thread function allocated the space for the return value dynamically then it is essential
for the higher level thread to free that space when it no longer needs the return value. If this
isn't done the program will leak memory.

\subsubsection*{Exercises}

\begin{enumerate}

\item Write a program that computes the square roots of the integers from 0 to 99 in a separate
  thread and returns an array of doubles containing the results. In the meantime the main thread
  should display a short message to the user and then display the results of the computation
  when they are ready.

\item Imagine that the computations done by the program above were much more time consuming than
  merely calculating a few square roots. Imagine also that displaying the "short message" was
  also fairly time consuming. For example, perhaps the message needed to be fetched from a
  network server as HTML and then rendered. Would you expect the multi-threaded program to
  perform better than a single threaded program that, for example, did the calculations first
  and then fetched the message? Explain.

\end{enumerate}



\section{Thread Synchronization}
\label{sec:thread-synchronization}

In order to effectively work together the threads in a program usually need to share information
or coordinate their activity. Many ways to do this have been devised and such techniques usually
go under the name of \newterm{thread synchronization}. In this section I will outline several
common methods of thread synchronization and show how they can be done using POSIX threads.

\subsection{Mutual Exclusion}
\label{subsec:mutual-exclusion}

When writing multi-threaded programs it is frequently necessary to enforce mutually exclusive
access to a shared data object. This is done with mutex objects. The idea is to associate a
mutex with each shared data object and then require every thread that wishes to use the shared
data object to first lock the mutex before doing so. Here are the particulars:

\begin{enumerate}

\item Declare an object of type \snippet{pthread\_mutex\_t}.

\item Initialize the object by calling \snippet{pthread\_mutex\_init()} or by using the special
  static initializer \snippet{PTHREAD\_\-MUTEX\_\-INITIALIZER}.
  
\item Call \snippet{pthread\_mutex\_lock()} to gain exclusive access to the shared data object.

\item Call \snippet{pthread\_mutex\_unlock()} to release the exclusive access and allow another
  thread to use the shared data object.

\item Get rid of the object by calling \snippet{pthread\_mutex\_destroy()}.

\end{enumerate}

The program of Listing~\ref{lst:mutex-example} demonstrates the basic approach. It is important
to understand that if a thread attempts to lock the mutex while some other thread has it locked,
the second thread is blocked until the first releases the mutex with
\snippet{pthread\_mutex\_unlock()}.

\begin{lstlisting}[float=tp,frame=single,xleftmargin=0in, caption={Mutex
Example},label=lst:mutex-example]
#include <pthread.h>
#include <unistd.h>

pthread_mutex_t lock;
int shared_data;
  // Usually shared data is more complex than just an int.

void *thread_function(void *arg)
{
  int i;

  for (i = 0; i < 1024*1024; ++i) {
    // Access the shared data here.
    pthread_mutex_lock(&lock);
    shared_data++;
    pthread_mutex_unlock(&lock);
  }
  return NULL;
}

int main(void)
{
  pthread_t thread_ID;
  void     *thread_result;
  int       i;

  // Initialize the mutex before trying to use it.
  pthread_mutex_init(&lock, NULL);

  pthread_create(&thread_ID, NULL, thread_function, NULL);

  // Try to use the shared data.
  for (i = 0; i < 10; ++i) {
    sleep(1);
    pthread_mutex_lock(&lock);
    printf("\rShared integer's value = %d\n", shared_data);
    pthread_mutex_unlock(&lock);
  }
  printf("\n");

  pthread_join(thread_ID, &thread_result);

  // Clean up the mutex when we are finished with it.
  pthread_mutex_destroy(&lock);
  return 0;
}
\end{lstlisting}

The code above uses dynamic initialization. However, it is also possible to initialize a mutex
object statically using the special symbol \snippet{PTHREAD\_\-MUTEX\_\-INITIALIZER} as the
initializer.

Be sure to observe these points:

\begin{enumerate}

\item No thread should attempt to lock or unlock a mutex that has not been initialized.

\item The thread that locks a mutex \emph{must} be the thread that unlocks it.

\item No thread should have the mutex locked when you destroy the mutex.

\end{enumerate}

In practice it is sometimes the case that threads are blocked on mutex objects when the program
wishes to terminate. In such a situation it might make sense to \snippet{pthread\_cancel()}
those threads before destroying the mutex objects they are blocked on. Coordinating this
properly can be tricky, however.

Notice that it is possible to assign special ``mutex attributes'' to a mutex object when it is
created. This is done by creating a mutex attribute object, assigning attributes to the object,
and then passing a pointer to the attribute object into \snippet{pthread\_mutex\_init()}. The
program in Listing~\ref{lst:mutex-example} just calls for default attributes by providing a
\snippet{NULL} pointer instead. In many cases this is perfectly adequate. The use of mutex
attribute objects is beyond the scope of this document.

It is important to understand that mutex locking is \newterm{advisory}. This means that no part
of the system (the operating system, the runtime system, or any other system) \emph{requires}
that you follow the rules. If you forget to lock a mutex before accessing the protected shared
data, your thread might interfere with the activity of another thread\ldots even a thread that
has played by the rules and locked the mutex first.

\subsubsection*{Exercises}

\begin{enumerate}
  
\item Enter the program in Listing~\ref{lst:mutex-example} and try it out. Does it behave the
  way you expected? Try different values for the maximum loop index in the thread function and
  different sleep times in the main function. Try removing the call to \snippet{sleep()}
  entirely. Try the program on different machines. Can you explain what is happening?

\item Suppose you are building a C++ string class that you intend to use in a multi-threaded
  program. You are worried about your string objects possibly getting corrupted if they are
  updated by more than one thread at a time. You consider adding a mutex as a member of each
  string and locking that mutex whenever any string method is called. Discuss the implications
  of this design. Be careful: this question is considerably trickier than it may appear!

\end{enumerate}

\subsection{Barriers}
\label{subsec:barriers}

A barrier is a synchronization primitive that forces multiple threads to reach the same point in
the program before any are allowed to continue. If a thread arrives at the barrier early it will
be suspended until the appropriate number of other threads arrive. Once the last thread has
reached the barrier, all the waiting threads are released and the threads can proceed
concurrently once again.

Barriers tend to be useful when multiple threads are executing different iterations of the same
loop. Often it is necessary to be sure all loop iterations are complete before moving on to a
task that requires the previous work to be entirely finished.

Listing~\ref{lst:barrier-example} shows a skeleton that illustrates how barriers can be used and
the context in which they might be useful. In this listing all threads execute the same thread
function but process different iterations of the \snippet{for} loop.

\begin{lstlisting}[float=tp,frame=single,xleftmargin=0in, caption={Barrier
Example},label=lst:barrier-example]
#include <pthread.h>

void *thread_function(void *arg)
{
  int i;
  int done = 0;
  struct TaskArg *thread_arg = (struct TaskArg *)arg;

  while (!done) {
    for (i = thread_arg->start; i < thread_arg->end; ++i) {
      // Work on loop iteration i
      // Each thread gets a separate TaskArg.
      // Different threads do different iterations.
    }

    if (pthread_barrier_wait(&loop_barrier) ==
          PTHREAD_BARRIER_SERIAL_THREAD) {

      // Prepare for next cycle.
      if (nothing_more) done = 1;
    }
    pthread_barrier_wait(&prep_barrier);
  }
  return NULL;
}
\end{lstlisting}

Two barriers are used. The first causes all threads to synchronize after the \snippet{for} loop
executes so that the preparation for the next cycle can be done knowing that the previous cycle
has fully completed. The barrier wait function returns the special value
\snippet{PTHREAD\_\-BARRIER\_\-SERIAL\_\-THREAD} in exactly one thread (chosen arbitrarily).
That thread is thus ``elected'' to take care of any serial work needed between cycles. In this
program there is no attempt to prepare for the next cycle in parallel.

The other threads immediately wait on the next barrier for the serial thread to catch up. Once
the preparation is fully completed, they are all released from the second barrier where they
loop back to do the next cycle of work.

This style of programming occurs frequently when writing truly parallel programs that expect the
threads to be physically executing at the same time. Such programs often use multiple threads to
complete different parts of the same task and use a barrier to be sure the complete task is
finished before allowing the higher level program logic to continue.

Barriers are instances of the \snippet{pthread\_barrier\_t} type. They are initialized with
\snippet{pthread\_barrier\_init} and cleaned up with \snippet{pthread\_barrier\_destroy}.
Listing~\ref{lst:barrier-main} shows a main function that uses the parallel looping function of
Listing~\ref{lst:barrier-example}.

\begin{lstlisting}[float=tp,frame=single,xleftmargin=0in, caption={Barrier
Example Main},label=lst:barrier-main]
pthread_barrier_t loop_barrier;
pthread_barrier_t prep_barrier;

int main(void)
{
  int i;
  pthread_t *thread_IDs;

  pthread_barrier_init(&loop_barrier, NULL, processor_count);
  pthread_barrier_init(&prep_barrier, NULL, processor_count);
  thread_IDs =
    (pthread_t *)malloc(processor_count * sizeof(pthread_t));

  // Create a thread for each CPU.
  for (i = 0; i < processor_count; ++i) {
    struct TaskArg *task =
      (struct TaskArg *)malloc(sizeof(struct TaskArg));
    task->start = i * iterations_per_processor;
    if (i == processor_count - 1)
      task->end = ITERATION_COUNT;
    else
      task->end = (i + 1) * iterations_per_processor;
    pthread_create(&thread_IDs[i],NULL,thread_function,task);
  }

  // Wait for threads to end.
  for (i = 0; i < processor_count; ++i) {
    pthread_join(thread_IDs[i], NULL);
  }

  free(thread_IDs);
  pthread_barrier_destroy(&loop_barrier);
  pthread_barrier_destroy(&prep_barrier);
  return 0;
}
\end{lstlisting}

The barrier objects themselves are global so they can be accessed by all threads. Alternatively
they could have been made local to the main function and passed into each thread indirectly (i.
e., with a pointer) by way of the thread's arguments. In any case, it is necessary for all
threads to see the same barrier object.

When the barrier objects are initialized it is necessary to provide the number of threads that
can accumulate on the barrier. For example, a barrier initialized with a count of five will
cause threads to wait until the fifth thread arrives. This count can't be changed once the
barrier object has been initialized.

This listing creates as many threads as there are processors using a previously defined variable
\snippet{processor\_count} (declaration not shown). Since the number of processors is unknown
when the program is written space for the thread IDs is allocated dynamically.

\subsubsection*{Exercises}

\textit{I need something here!}

\subsection{Condition Variables}
\label{subsec:condition-vars}

If you want one thread to signal an event to another thread, you need to use \newterm{condition
  variables}. The idea is that one thread waits until a certain condition is true. First it
tests the condition and, if it is not yet true, calls \snippet{pthread\_cond\_wait()} to block
until it is. At some later time another thread makes the condition true and calls
\snippet{pthread\_cond\_signal()} to unblock the first thread.

Every call to \snippet{pthread\_cond\_wait()} should be done as part of a conditional statement.
If you aren't doing that, then you are most likely using condition variables incorrectly. For
example

\begin{lstlisting}
if (flag == 0) pthread_cond_wait(...);
\end{lstlisting}

Here I'm waiting until the flag is not zero. You can test conditions of any complexity. For
example

\begin{lstlisting}
x = f(a, b);
if (x < 0 || x > 9) pthread_cond_wait(...);
\end{lstlisting}

Here I'm waiting until \snippet{x} is in the range from zero to nine inclusive where \snippet{x}
is computed in some complex way. Note that \snippet{pthread\_cond\_wait()} is only called if the
condition is not yet true. If the condition is already true, \snippet{pthread\_cond\_wait()} is
not called. This is necessary because condition variables do not remember that they have been
signaled.

If you look at my examples, you will see that there is a serious race condition in them. Suppose
the condition is not true. Then suppose that after the condition is tested but before
\snippet{pthread\_cond\_wait()} is called, the condition becomes true. The fact that the
condition is signaled (by some other thread) will be missed by \snippet{pthread\_cond\_wait()}.
The first thread will end up waiting on a condition that is already true. If the condition is
never signaled again the thread will be stuck waiting forever.

To deal with this problem, every time you use a condition variable you must also use a mutex to
prevent the race condition. For example:

\begin{lstlisting}
pthread_mutex_lock(&mutex);
x = f(a, b);
if (x < 0 || x > 9) pthread_cond_wait(&condition, &mutex);
pthread_mutex_unlock(&mutex);
\end{lstlisting}

The thread that signals this condition will use the same mutex to gain exclusive access to the
whatever values are involved in computing the condition (which depends on what function
\snippet{f} does in this example). Thus there is no way that the signaling could occur between
the test of the condition and the waiting on the condition.

For the above to work, \snippet{pthread\_cond\_wait()} needs to wait on the condition and unlock
the mutex as an atomic action. It does this, but it needs to know which mutex to unlock. Hence
the need for the second parameter of \snippet{pthread\_cond\_wait()}. When the condition is
signaled, \snippet{pthread\_cond\_wait()} will lock the mutex again before returning so that the
\snippet{pthread\_mutex\_unlock()} in the above example is appropriate regardless of which
branch of the if is taken.

Here is how the signaling thread might look

\begin{lstlisting}
pthread_mutex_lock(&mutex);
a = ...
b = ...
x = f(a, b);
if (x >= 0 && x <= 9) pthread_cond_signal(&condition);
pthread_mutex_unlock(&mutex);
\end{lstlisting}

Before doing a computation that might change the condition, the signaling thread locks the mutex
to make sure the waiting thread can't get caught in a race condition. For example, in this case
it wouldn't do if the \emph{waiting} thread saw a new version of \snippet{a} but the old version
of \snippet{b}. In that case it might calculate an inappropriate value of \snippet{f(a, b)} and
wait when it shouldn't.

There is a further subtlety regarding the use of condition variables. In certain situations the
wait function might return even though the condition variable has not actually been signaled.
For example, if the Unix process in general receives an operating system signal, the thread
blocked in \snippet{pthread\_cond\_wait()} might be elected to process the signal handling
function. If system calls are not restarting (the default in many cases) the
\snippet{pthread\_cond\_wait()} call might return with an interrupted system call error
code\footnote{Of course this assumes you are dealing with an actual kernel thread. If the thread
  is purely a user mode thread such unexpected returns won't occur.}. This has nothing to do
with the state of the condition so proceeding as if the condition is true would be
inappropriate.

The solution to this problem is to simply retest the condition after
\snippet{pthread\-\_cond\-\_wait()} returns. This is most easily done using a loop. For example

\begin{lstlisting}
pthread_mutex_lock(&mutex);
while (1) {
  x = f(a, b);
  if (x < 0 || x > 9) pthread_cond_wait(&condition, &mutex);
    else break;
}
pthread_mutex_unlock(&mutex);
\end{lstlisting}

Of course this assumes you want to ignore any spurious returns from the wait function. In a more
complex application you might want to process the error codes in various ways depending on the
situation.

The \snippet{pthread\_cond\_signal} function releases only one thread at a time. In some cases
it is desirable to release all threads waiting on a condition. This can be accomplished using
\snippet{pthread\_cond\_broadcast}. For example

\begin{lstlisting}
pthread_mutex_lock(&mutex);
a = ...
b = ...
x = f(a, b);
if (x >= 0 && x <= 9) pthread_cond_broadcast(&condition);
pthread_mutex_unlock(&mutex);
\end{lstlisting}

The example in Listing~\ref{lst:cond-example} illustrates the use of condition variables in the
context of a program. Although contrived, this example is at least complete and compilable.

\begin{lstlisting}[float=tp,frame=single,xleftmargin=0in,
  caption={Condition Variable Example},label=lst:cond-example]
#include <pthread.h>
#include <unistd.h>

pthread_cond_t  is_zero;
pthread_mutex_t mutex;   // Condition variables needs a mutex.
int shared_data = 32767; // Or some other large number.

void *thread_function(void *arg)
{
  // Imagine doing something useful.
  while (shared_data > 0) {
    // The other thread sees the shared data consistently.
    pthread_mutex_lock(&mutex);
    --shared_data;
    pthread_mutex_unlock(&mutex);
  }
  
  // Signal the condition.
  pthread_cond_signal(&is_zero);
  return NULL;
}

int main(void)
{
  pthread_t thread_ID;
  void     *exit_status;
  int       i;

  pthread_cond_init(&is_zero, NULL);
  pthread_mutex_init(&mutex, NULL);

  pthread_create(&thread_ID, NULL, thread_function, NULL);

  // Wait for the shared data to reach zero.
  pthread_mutex_lock(&mutex);
  while (shared_data != 0)
    pthread_cond_wait(&is_zero, &mutex);
  pthread_mutex_unlock(&mutex);

  pthread_join(thread_ID, &exit_status);

  pthread_mutex_destroy(&mutex);
  pthread_cond_destroy(&is_zero);
  return 0;
}
\end{lstlisting}

Notice that in this program the condition variables are also initialized and destroyed by calls
to appropriate functions. As with mutex variables you can also initialize condition variables
statically using a special symbol: \snippet{PTHREAD\_\-COND\_\-INITIALIZER}.

\subsubsection*{Exercises}

\begin{enumerate}
  
\item Modify the program in Listing~\ref{lst:cond-example} to print messages and add delays (or
  wait for user input) at various places so you can verify that the thread is, in fact, waiting
  for the condition as appropriate. Verify that the thread does not wait if the condition is
  already true when it is first tested.

\item In the text above, when a condition is signaled the signaling thread calls
  \snippet{pthread\_cond\_signal()} before unlocking the mutex. However, it is also possible
  swap those operations as shown below.

\begin{lstlisting}
pthread_mutex_lock(&mutex);
a = ...
b = ...
x = f(a, b);
pthread_mutex_unlock(&mutex);
if (x >= 0 && x <= 9) pthread_cond_signal(&condition);
\end{lstlisting}

Does this result in the same behavior as before? Are any race conditions introduced (or fixed)
by this change? How does this approach impact application performance?

\end{enumerate}

\subsection{Semaphores}
\label{subsec:semaphores}

Semaphore are essentially glorified integer counters. They support two primary operations. One
operation, called \newterm{down} or \newterm{wait}, attempts to decrement the counter. The other
operation, called \newterm{up}, \newterm{signal}, or \newterm{post} attempts to increment the
counter. What makes semaphores special is that if a thread tries to wait on a zero semaphore it
is blocked instead. Later when another thread posts the semaphore the blocked thread is
activated while the semaphore remains at zero. In effect, the posting causes the semaphore to be
incremented but then the thread that was blocked trying to do a wait is allowed to proceed,
causing the semaphore to be immediately decremented again.

If multiple threads are blocked waiting on a semaphore then the system chooses one to unblock.
Exactly how this choice is made is generally system dependent. You can not assume that it will
be in FIFO order\footnote{If threads have different priorities, normally the highest priority
  thread is allowed to go first.}. However, the order in which the threads are unblocked is not
normally a concern. If it is, then your program may not be very well designed.

A semaphore with an initial value of one can be used like a mutex. When a thread wishes to enter
its critical section and access a shared data structure, it does a wait operation on the
semaphore. If no other thread is in its critical section, the semaphore will have its initial
value of one and the wait will return immediately. The semaphore will then be zero. If another
thread tries to wait on the semaphore during this time it will be blocked. When the first thread
is finished executing its critical section it does a post operation on the semaphore. This will
unblock one waiting thread or, if there are no waiting threads, increment the semaphore back to
its initial value of one. A semaphore used in this way is called a \newterm{binary semaphore}
because it has exactly two states.

However, because semaphores are integers they can take on values larger than one. Thus they are
often used to count scarce resources. For example a thread might wait on a semaphore to
effective reserve one item of a resource. If there are no items left, the semaphore will be zero
and the wait operation will block. When a thread is finished using an item of a resource it
posts the semaphore to either increment the count of available items or to allow a blocked
thread to access the now available item. A semaphore used in this way is called a
\newterm{counting semaphore}.

The POSIX semaphore API is not really part of the normal pthread API. Instead POSIX standardizes
semaphores under a different API. Traditional Unix systems support shared memory, message
queues, and semaphores as part of what is called ``System V Interprocess Communication'' (System
V IPC). POSIX also provides shared memory, message queues, and semaphores as a package that
competes with, or replaces, the older standard. The functionality of the two systems is similar
although the details of the two APIs are different.

Note that POSIX semaphores, like System V IPC semaphores, can be used to synchronize two or more
separate processes. This is different than pthread mutexes. A mutex can only be used by threads
in the same process. Because POSIX semaphores can be used for interprocess communication, they
have the option of being named. One process can create a semaphore under a particular name and
other processes can open that semaphore by name. In this tutorial, however, I will focus only on
synchronizing threads in the same process.

\begin{lstlisting}[float=tp,frame=single,xleftmargin=0in,
  caption={Semaphore Example},label=lst:semaphore-example]
#include <semaphore.h>

int   shared;
sem_t binary_sem;  // Used like a mutex.

void *thread_function(void *arg)
{
  sem_wait(&binary_sem);    // Decrements count.
  // Used shared resource.
  sem_post(&binary_sem);    // Increments count.
}

void main(void)
{
  sem_init(&binary_sem, 0, 1); // Initial count of 1.
  
  // Start threads here.
  
  sem_wait(&binary_sem);
  // Use shared resource.
  sem_post(&binary_sem);
  
  // Join with threads here.
  
  sem_destroy(&binary_sem);
  return 0;
}
\end{lstlisting}

The skeleton program in Listing~\ref{lst:semaphore-example} shows how to initialize, clean up,
and use a POSIX semaphore. For brevity the skeleton program does not show the threads being
created or joined nor does it show any error handling. See the manual pages for the various
functions for more information on error returns.

Another difference between a pthread mutex and a semaphore is that, unlike a mutex, a semaphore
can be posted in a different thread than the thread that does the wait operation. This is
necessary when using a semaphore to count instances of a scarce resource. The skeleton program
in Listing~\ref{lst:semaphore-example} is using a semaphore like a mutex. I did this to simplify
the listing so that the functions used to manipulate the semaphore would be clear.

To see semaphores being used in a more interesting way, consider the classic producer/consumer
problem. In this problem one thread is producing items (say, objects of type \snippet{void *}
that might be pointing at other objects of arbitrary complexity) while another thread is
consuming those items. Listing~\ref{lst:pcbuffer-header} shows an abstract data type that
implements a buffer that can be used to hold items as they pass from one thread to another.

\begin{lstlisting}[float=tp,frame=single,xleftmargin=0in,
  caption={Producer/Consumer Abstract Type},label=lst:pcbuffer-header]
#ifndef PCBUFFER_H
#define PCBUFFER_H

#include <pthread.h>
#include <semaphore.h>

#define PCBUFFER_SIZE 8

typedef struct {
  void *buffer[PCBUFFER_SIZE];
  pthread_mutex_t lock;
  sem_t           used;
  sem_t           free;
  int             next_in;   // Next available slot.
  int             next_out;  // Oldest used slot.
} pcbuffer_t;

void  pcbuffer_init(pcbuffer_t *);
void  pcbuffer_destroy(pcbuffer_t *);
void  pcbuffer_push(pcbuffer_t *, void *);
void *pcbuffer_pop(pcbuffer_t *);

#endif
\end{lstlisting}

The solution actually needs two semaphores. One is used to count the number of free slots in the
buffer and the other is used to count the number of used slots. This is necessary because we
must block the producer when the buffer is full and we must block the consumer when the buffer
is empty. However, semaphores only block their caller when one attempts to decremented them
below zero; they never block when they are incremented.
Listing~\ref{lst:pcbuffer-implementation} shows the implementation in detail.

\begin{lstlisting}[float=tp,frame=single,xleftmargin=0in,
  caption={Producer/Consumer Implementation},label=lst:pcbuffer-implementation]
#include "pcbuffer.h"

void pcbuffer_init(pcbuffer_t *p) {
  pthread_mutex_init(&p->lock);
  sem_init(&p->used, 0, 0);
  sem_init(&p->free, 0, PCBUFFER_SIZE);
  p->next_in  = 0;
  p->next_out = 0;
}

void pcbuffer_destroy(pcbuffer_t *p) {
  pthread_mutex_destroy(&p->lock);
  sem_destroy(&p->used);
  sem_destroy(&p->free);
}

void pcbuffer_push(pcbuffer_t *p, void *value) {
  sem_wait(&p->free);
  pthread_mutex_lock(&p->lock);
  p->buffer[p->next_in++] = value;
  if (next_in == PCBUFFER_SIZE) next_in = 0;
  pthread_mutex_unlock(&p->lock);
  sem_post(&p->used);
}

void *pcbuffer_pop(pcbuffer_t *p) {
  void *return_value;

  sem_wait(&p->used);
  pthread_mutex_lock(&p->lock);
  return_value = p->buffer[p->next_out++];
  if (next_out == PCBUFFER_SIZE) next_out = 0;
  pthread_mutex_unlock(&p->lock);
  sem_post(&p->free);
  return return_value;
}
\end{lstlisting}

The initialization and clean-up functions are straight forward. In contrast the push and pop
functions are surprisingly subtle. In each it is necessary to first reserve a unit of the
limited resource. In the case of \snippet{pcbuffer\_push()} we must reserve a free slot. If no
free slots are available, the call to \snippet{sem\_wait(\&p->free)} will block until the
consumer posts that semaphore after removing an item.

Once a slot has been reserved we lock the buffer to ensure that no other thread can corrupt it
by modifying it at the same time. Finally, after unlocking we post the other semaphore to
perhaps unblock the other thread as necessary. For example the call to
\snippet{sem\_signal(\&p->used)} will unblock a waiting consumer to handle the item just stored
in the buffer.

\subsubsection*{Exercises}

\begin{enumerate}

\item Using POSIX mutex and condition variables, implement a semaphore abstract type. For
  example, consider a header file containing the following.

\begin{lstlisting}
typedef struct {
  // Fill in members as appropriate.
} semaphore_t;

void semaphore_init(
  struct semaphore *s, int initial_count);
void semaphore_destroy(semaphore_t *s);
void semaphore_up(semaphore_t *s);
void semaphore_down(semaphore_t *s);
\end{lstlisting}

Implement the functions declared above. This shows that semaphores are not strictly necessary as
part of a low level API.

\item Some semaphore APIs (such as the Win32 API) allow the post operation to advance the value
  of a semaphore by more than one. This can be implemented by executing a basic post operation
  multiple times in a loop. However, such an approach is inefficient if the number to be added
  to the semaphore is large. Extend your solution for the question above so that
  \snippet{semaphore\_up} takes an additional integer parameter specifying the how much the
  semaphore value is to be advanced. Try to use an efficient method of handling large advances.
  Make sure your solution works properly and does not suffer from any race conditions even when
  there are multiple threads waiting on the semaphore.

\end{enumerate}

\subsection{Reader/Writer Locks}
\label{subsec:reader-writer}

Mutex objects provide mutually exclusive access to a shared resource. But sometimes complete
mutual exclusion is unnecessarily restrictive. If two threads are only interested in reading a
shared resource, it should be possible to allow both to access the resource at the same time. If
neither thread tries to modify the resource, the resource will never be in an inconsistent state
and simultaneous access is safe. Indeed, it is common for there to be multiple threads trying to
read a shared resource where updates to that resource are uncommon. For example a tree data
structure might be used many times by multiple threads to look up information and yet updated
only rarely by a single thread.

To support this usage POSIX provides \newterm{reader/writer locks}. Multiple readers can lock
such an object without blocking each other, but when a single writer acquires the lock it has
exclusive access to the resource. All following readers or writers will block as long as a
writer holds the lock.

The skeleton program in Listing~\ref{lst:rwlock-example} shows the basic structure. By now the
pattern of initialization, destruction, and use should look familiar. In a more typical program
the thread function where the read lock is acquired might be executed by many threads while the
main function where the write lock is needed might be executed by only one thread. Notice in
this case that the same function is used to unlock both read locks and write locks.

\begin{lstlisting}[float=tp,frame=single,xleftmargin=0in,
  caption={Reader/Writer Lock Example},label=lst:rwlock-example]
#include <pthread.h>

int              shared;
pthread_rwlock_t lock;

void *thread_function(void *arg)
{
  pthread_rwlock_rdlock(&lock);
  // Read from the shared resource.
  pthread_rwlock_unlock(&lock);
}

void main(void)
{
  pthread_rwlock_init(&lock, NULL);
  
  // Start threads here.
  
  pthread_rwlock_wrlock(&lock);
  // Write to the shared resource.
  pthread_rwlock_unlock(&lock);
  
  // Join with threads here.
  
  pthread_rwlock_destroy(&lock);
  return 0;
}
\end{lstlisting}

Depending on the implementation, a steady stream of readers might permanently lock out a writer.
This situation is called \newterm{writer starvation}. On the other hand if the implementation
favors writers in the sense of letting waiting writers obtain the lock as soon as possible,
\newterm{reader starvation} may occur. The POSIX standard favors writers, depending on specific
thread priorities. This behavior is reasonable because writers are presumed to be rare and the
updates they want to do are presumed to be important. It is more realistic to expect a steady
stream of readers than a steady stream of writers. Thus if readers were favored, writer
starvation would be a significant concern.

It is important to note that the system does not (can not) enforce the read-only restriction on
reader threads. There is nothing to stop a thread from acquiring a read lock and then go ahead
and write to the shared resource anyway. If multiple reader threads do this, data corruption
might occur. It is the programmer's responsibility to ensure this does not happen.

\subsubsection*{Exercises}

\begin{enumerate}

\item Implement a reader/writer lock abstract type in terms of other POSIX synchronization
  primitives. Can your implementation cause reader or writer starvation?

\end{enumerate}

\subsection{Monitors}
\label{sec:monitors}

The synchronization primitives discussed so far are all fairly primitive. This makes them
flexible but it also makes them difficult to use properly. A synchronization abstraction that is
commonly provided by programming languages with direct support for concurrency is the
\newterm{monitor}. Essentially a monitor is an encapsulation of data and code with the property
that only one thread at a time can be inside the monitor. Thus mutual exclusion is provided
automatically by the monitor construct.

The POSIX thread API does not support monitors directly but because it is a useful facility
there is value in exploring how a monitor could be implemented with the POSIX thread API. Since
C is a relatively primitive language, the syntactic features for declaring and using monitors
are not directly available. Instead the programmer must adhere to certain programming
conventions. This is typical of programming in the C environment.

Listing~\ref{lst:basic-monitor} illustrates the general approach. The monitor maps to a single
translation unit (source file) where the internal data and support functions have internal
linkage (marked as \snippet{static}). A master mutex object is used to control access to the
monitor. The monitor operations are ordinary external functions with the property that they lock
the mutex on entry and unlock it on exit. Care must be taken by the programmer to ensure that
the mutex is properly unlocked on every exit path from those functions. In addition, the
external functions can't call each other without risking deadlock on the monitor
mutex\footnote{Unless the monitor mutex is a recursive mutex.}.

\begin{lstlisting}[float=tp,frame=single,xleftmargin=0in,
  caption={Basic Monitor},label=lst:basic-monitor]
#include "service.h"

static pthread_mutex_t monitor_lock =
  PTHREAD_MUTEX_INITIALIZER;

static int internal_data;

static void internal_function( void ) { ... }

void service_function( void )
{
    pthread_mutex_lock(&monitor_lock);
    ...
    // Use internal data and functions freely.
    ...
    pthread_mutex_unlock(&monitor_lock);
}
\end{lstlisting}

Associated with the monitor is a header file that declares any externally visible service
functions and the types they require. Multiple threads can call these service functions safely.
Only one thread at a time is allowed inside the monitor so the internal data will never
experience simultaneous updates.

In the case where some service functions only read the internal data, it may make sense to use a
POSIX reader/writer lock as the monitor lock. Functions that only read the data can obtain a
read lock on the monitor, allowing several threads to call such functions simultaneously. Of
course functions that update the internal data will need to obtain a write lock on the monitor.

Unfortunately mutual exclusion inside the monitor is not enough to make the monitor construct
generally useful. In many situations it is necessary for a thread to suspend itself inside the
monitor while waiting for a certain condition to be true (for example, for data to be ready,
etc). Naturally the monitor must be unlocked when the thread is suspended so that another thread
is allowed inside the monitor. If this were not done, the suspended thread would wait forever
for a condition that could never arise.

POSIX condition variables are a good fit for these semantics. The suspending thread can call
\snippet{pthread\_cond\_wait} on an internal condition variable, passing the monitor lock as the
second parameter. This will suspend the thread and unlock the monitor in an atomic manner. When
another thread signals the condition, the first thread will attempt to reacquire the mutex
before continuing. It will be prevented from doing this until the signaling thread leaves the
monitor (since the signaling thread will still hold the monitor mutex). Thus the rule that only
a single thread executes in the monitor at a time is enforced. Listing~\ref{lst:monitor-example}
illustrates the approach.

\begin{lstlisting}[float=tp,frame=single,xleftmargin=0in,
  caption={Monitor Example},label=lst:monitor-example]
#include "service.h"

static pthread_mutex_t monitor_lock =
  PTHREAD_MUTEX_INITIALIZER;
static pthread_cond_t condition =
  PTHREAD_COND_INITIALIZER;

void service_function1( void )
{
    pthread_mutex_lock(&monitor_lock);
    ...
    while (!data_ready( )) {
      pthread_cond_wait(&condition, &monitor_lock);
    }
    ...
    pthread_mutex_unlock(&monitor_lock);
}

void service_function2( void )
{
    pthread_mutex_lock(&monitor_lock);
    ...
    make_data_ready( );
    pthread_cond_signal(&condition);
    ...
    pthread_mutex_unlock(&monitor_lock);
}
\end{lstlisting}

In this example the functions \snippet{data\_ready} and \snippet{make\_data\_ready} are assumed
to be internal monitor functions (not shown in Listing~\ref{lst:monitor-example}). Notice that
it is important to test the condition in a loop when calling the wait operation. This is to
protect against spurious wake-ups (for example due to operating system signals). It also
protects against another problem: When the signaling thread leaves the monitor, some other
thread waiting to enter the monitor might acquire the mutex before the awakened thread does so.
This other thread might then invalidate the condition before the awakened thread is able to
return from \snippet{pthread\_cond\_wait}. Thus retesting the condition before continuing from
the wait is a must.

\subsubsection*{Exercises}

\begin{enumerate}

\item POSIX conditional waits take a pointer to a mutex object. However, if one wants to use a
  reader/writer lock to control access to the monitor, \snippet{pthread\_cond\_wait} can't be
  used directly. How can this be handled?

\end{enumerate}


\section{Thread Models}
\label{sec:thread-models}

In this section I will describe some ways that threads can be used in real programs. The goal is
to give you a feeling for the kind of design ideas that lend themselves to a threaded solution.
It is usually possible to build a single threaded program that solves the same problem, but in
some cases the single threaded solution is awkward and difficult to manage. Be aware, however,
that single threaded solutions are often the most appropriate. There can be a significant amount
of overhead associated with synchronizing threads; multiple threads, poorly applied, can easily
result in a slower and more confusing program.

\subsection{Boss/Worker Model}
\label{subsec:bossworker-model}

The idea in the boss/worker model is to have a single \newterm{boss} thread that creates work
and several \newterm{worker} threads that process the work. Typically the boss thread creates a
certain number of workers right away---even before any work has arrived. The worker threads form
a \newterm{thread pool} and are all programmed to block immediately. When the boss thread
generates some work, it arranges to have one worker thread unblock to handle it. Should all
workers be busy the boss thread might:

\begin{enumerate}
  
\item Queue up the work to be handled later as soon as a worker is free.

\item Create more worker threads.

\item Block until a worker is free to take the new work.

\end{enumerate}

If no work has arrived recently and there are an excessive number of worker threads in the
thread pool, the boss thread might terminate a few of them to recover resources. In any case,
since creating and terminate threads is relatively expensive (compared to, say, blocking on a
mutex) it is generally better to avoid creating a thread for each unit of work produced.

You have already seen this model in action many times. Consider a bank. When you arrive you have
work that needs doing. You get in a queue and wait for a free teller (worker thread). When a
teller is available that teller handles your work while other tellers are handling other work at
the same time. Should someone in line have an unusually complicated transaction, it won't hold
up the line. Only one teller will be tied up dealing with the large work item. The other tellers
will be available to handle other people's work normally. Thus the response time is reasonable
even when some work items are very time consuming.

A web server is another excellent example of where the boss/worker model can be used. The boss
thread listens to the network for incoming connections. When a connection is made, the boss
thread directs a worker thread to handle that connection. The boss thread then returns to
listening to the network again. In this way many connections can be handled at once. If a
particularly time consuming connection is active, it won't prevent the server for dealing with
other connections as well.

This model works the best when the work items are independent of each other. If the work items
depend on each other or have to be processed in a particular sequence the worker threads have to
talk to each other and the overall efficiency of this model is much reduced. Also, if you run a
boss/worker program on a single processor machine it is important that servicing a work item
involves a fair amount of blocking. If the work items are all 100\% CPU bound then there won't
be any performance enhancement. A single thread servicing all the items in sequence would be
just as fast as having multiple threads servicing several items at once. However, if servicing
an item requires a lot of blocking, or if multiple CPUs are available, then another thread can
use the CPU while the first is blocked and the overall performance is better (often drastically
so).

\subsection{Pipeline Model}
\label{subsec:pipeline-model}

Many programs take some input, transform it in several stages, and then output the result.
Instead of having a single thread perform each step in sequence you could have a separate thread
handling each stage. The result is much like an assembly line. The data flows from one worker to
another and each worker performs their particular operation on the data. By the time the data
reaches the end of the line it has been fully transformed into the desired output.

Usually writing a single threaded program to process sequential data in stages is fairly
straightforward. However, a multithreaded pipeline has the potential to outperform the single
threaded solution. In general, if there are $N$ stages to the pipeline there can be $N$ items
being operated on at once by the multithreaded program and the result will be $N$ times faster.
In practice it rarely works out this well. To obtain its full efficiency the time required for
every stage must be identical and the processing of one stage can't in any way slow down the
processing of the others. If the program runs on a single processor the operations being done in
each stage must block frequently so that the CPU can execute another stage while the blocked
stages are waiting (for example, for I/O).

To balance the load between the stages, the programmer might need to use profiling tools to find
out the relative time used by the different stages. Stages that are very short can be combined
with the stage on either side while stages that are very long can be split into multiple stages
(ideally with blocking operations divided evenly between the stages). Getting this balance just
right is difficult yet without it the multithreaded solution to the pipeline model will hardly
be any faster than the single threaded solution. In fact, because of locking overhead, it may
even be slower\footnote{The buffers between the stages must be careful to avoid race conditions
  and overflow/underflow conditions. This involves a significant amount of locking activity.}.

\subsection{Background Task Model}
\label{subsec:background-model}

Many programs have tasks that they would like to complete ``in the background.'' For example a
program might want to backup its data files every 10 minutes or update a special status window
every 5 seconds. It is awkward to program such things with only a single thread. The program
must remember to check the time regularly and call the background function whenever an
appropriate amount of time has elapsed. Since that might happen at any point in the program's
execution, the program's logic must be littered with calls to functions that are largely
unrelated to the main flow of the program.

With multiple threads, however, this model is quite easy to program. A background thread can be
created when the program initializes itself. That thread can sleep for a certain fixed time and
then, when it wakes up, perform the necessary background operation. The thread would then just
loop back and sleep again until the next time interval has expired. This can happen
independently of the main program's logic. The main complication involved with programming this
model is in making sure that the background thread is properly synchronized with the main thread
when they access shared data.

In this approach the threads are used in a manner similar to the way interrupt service routines
are used. They provide background services that the main program does not have to explicity
invoke. Many useful tasks can be effectively handled in this way.

\subsection{Interface/Implementation Model}
\label{subsec:interface-model}

Most graphical environments are event driven. Each action taken by the user is a separate event
that the program must handle. Examples of events include mouse clicks, menu selections,
keystrokes, and so forth. Typically the program contains a single function that is called by the
system whenever an event happens. That function must process the event and then return before
the system calls the function with the next event. If the event handling function does not
return quickly enough events will back up and the user will perceive the program as unresponsive
and sluggish. In an extreme case the program will even appear to be dead.

To avoid this, the program can use multiple threads. If handling an event is going to be time
consuming and difficult (and involve a lot of blocking), the event handling function can just
create a thread to deal with it and then return at once. This gives the event handling function
the opportunity to handle additional events while past events are being processed by other
threads. The result is that the program's interface remains responsive even if some of the
operations requested are very time consuming.

It is not necessary for an entire program to be organized this way. Internal modules in a
program can use the same trick. When a function is called in such a module, the function might
create a thread to carry out the requested operation and then return at once. The calling
program will see the function as very quick and responsive even though the actual work requested
hasn't really been completed when the function returns.

The difficulty with this model is that eventually the user of an interface will need to know for
sure when certain operations requested in the past have actually completed. Some way of
coordinating that information must be provided. Also it is difficult for the program to report
errors effectively with this model because an error might occur long after the operation was
requested and apparently handled.

Many operating systems themselves use this model extensively. For example, when a program writes
to a file, the file is typically not put onto the disk at once. Instead it is just put into a
cache (faster) and written to disk latter when the system is less busy. In effect, the operating
system writes to disk in a separate thread that is independent of the thread that actually
requested the write operation in the first place.

\subsection{General Comments}
\label{subsec:general-model}

In general, multithreaded programs work best if the threads are as independent as possible. The
less the threads have to communicate with each other the better. Whenever threads have to
synchronize or share data there will be locking overhead and time spent waiting for other
threads. Time blocked while waiting for another thread is time wasted. Such a thread is not
doing anything useful. In contrast, if a thread is blocked waiting for I/O it is doing something
that the program needs done. Such blocking is good because it allows other threads to get the
CPU. But if a thread waits for another thread then it is not accomplishing anything that the
program needs. The more threads interact with each other the more time they will spend waiting
for each other and the more inefficient the program will be.

It is easy to understand this idea when you think about working with another person on a common
project. If you and your partner can do two largely independent activities you can both work
without getting in each other's way and you can get twice as much work done. But if you try to
work too closely then one of you will simply be waiting for the other and the work won't get
done any more quickly than it would by a single person alone. Consider what would happen if you
decided to type a paper with your partner but that you and your partner had to alternate
keystrokes on the keyboard. To type ``Hello'' first you type `H' then your partner types `e'
then you type `l' and so on. Obviously this would be very inefficient. You would spend more time
getting the alternation right than you would actually typing keys. The exact same issues arise
in multithreaded programs. An improperly written multithreaded program can be slower---sometimes
a lot slower---than its single threaded equivalent.

If two tasks are very independent, they can often be handled by two entirely separate processes.
Large software systems are often composed of many executable files, each taking care of a single
aspect of the system. At this level the system is ``multithreaded'' even if the individual
programs are not. However, it can be difficult for multiple processes to share information
effectively. Putting multiple threads into a single process makes the parallelism more
\newterm{fine grained}, allows the threads to interact more closely, and share more resources.
This can be a good thing. But if taken to extreme it causes inefficiencies as I described above.
A good multithreaded program will strike the right balance between sharing and independence.
That balance is often difficult to find.


\section{Thread Safety}
\label{sec:thread-safety}

Typically when a complicated object is operated on, it goes through several intermediate,
invalid states before the operation is complete. As an analogy consider what a surgeon does when
he operates on a person. Although the purpose of the operation is to increase the patient's
health, the surgeon performs several steps that would greatly decrease the patient's health if
they were left incomplete! Similarly a function that operates on an object will often temporary
put that object into an unusable state while it performs the update. Once the update is
complete, the function (if written correctly) will leave the object is a fully functional state
again.

Should another thread try to use an object while it is in an unusable state (often called an
\newterm{inconsistent state}) the object will not respond properly and the result will be
undefined. Keeping this from happening is the essential problem of thread safety. The problem
doesn't come up in a single threaded program because there is no possibility of another thread
trying to access the object while the first thread is updating it\footnote{Unless exceptions are
  a possibility. In that case the updating thread might abort the update and then later try to
  access the incompletely updated object. This causes the same sort of problems to occur.}.

\subsection{Levels of Thread Safety}
\label{subsec:levels-safety}

People often have problems discussing thread safety because there are many different levels of
safety one might want to talk about. Just saying that a piece of code is ``thread safe'' doesn't
really say all that much. Yet most people have certain natural expectations about thread safety.
Sometimes those expectations are reasonable and valid, but sometimes they are not. Here are some
of those expectations.

\begin{itemize}
  
\item Reading an object's value with multiple threads is not normally expected to be a problem.
  Problems only occur when an object is updated since it is only then that it has to be modified
  and run the risk of entering inconsistent states.
  
  \emph{However} some objects have internal state that gets modified even when its value is read
  (think about an object that has an internal cache). If two threads try to read such an object
  there might be problems unless the read operations on that object have been designed to handle
  the multiple threads properly.
  
\item Updating two independent objects, even of the same type, is not normally expected to be a
  problem. It is usually assumed that objects that appear to be independent are, in fact,
  independent and thus the inconsistent states of one such object have no impact on the other.
  
  \emph{However} some objects share information behind the scenes (static class data, global
  data, etc) that causes them to be linked internally even when they do not appear to be linked
  from a logical point of view. In that case, modifying two ``independent'' objects might cause
  a problem anyway. Consider:

\begin{lstlisting}
void f()                    void g()
{                           {
  std::string x;              std::string y;

  // Modify x.                // Modify y.
}                           }
\end{lstlisting}
  
If one thread is in function \snippet{f()} modifying the string \snippet{x} and another is in
function \snippet{g()} modifying string \snippet{y}, will there be a problem? Most of the time
you can assume that the two apparently independent objects can be simultaneously modified
without synchronization. But it is possible, depending on how \snippet{std::string} is
implemented, that the two objects share some data internally and that simultaneous modifications
\emph{will} cause a problem. In fact, even if one of the functions merely reads the value of the
string, there might be a problem if they share internal data that is being updated by the other
function.
  
\item Functions that acquire resources, even if from a common pool of resources, are not
  normally expected to be a problem. Consider:

\begin{lstlisting}
void f()                    void g()
{                           {
  char *p = new char[512];    char *p = new char[512];

  // Use the array p.         // Use the array p.
}                           }
\end{lstlisting}
  
If one thread is in function \snippet{f()} and another thread is in function \snippet{g()}, both
threads might try to allocate memory simultaneously by invoking the \snippet{new} operator. In a
multi-threaded environment, it is safe to assume that \snippet{new} has been written to work
correctly in this case even though both invocations of \snippet{new} are trying to take memory
from the same pool of memory. Internally \snippet{new} will synchronize the threads so that each
call will return a unique allocation and the internal memory pool will not be corrupted. Similar
comments can be made about functions that open files, make network connections, and perform
other resource allocation tasks.
  
\emph{However} if the resource allocation functions are not designed with threads in mind then
they may well fail if invoked by multiple threads at once.

\end{itemize}

What people typically expect to cause problems is when a program tries to access (update or
read) an object while another thread is updating that same object. Global objects are
particularly prone to this problem. Local objects are much less so. For example:

\begin{lstlisting}
std::string x;

void f()
{
  std::string y;

  // Modify x and y.
}
\end{lstlisting}

If two threads enter function \snippet{f()} at the same time, they will get different versions
of the string \snippet{y}. This is because every thread has its own stack and local objects are
allocated on the thread's stack. Thus every thread has its own, independent copy of the local
objects. As a result, manipulating \snippet{y} inside \snippet{f()} will not cause a problem
(assuming that manipulating independent objects is safe). However, since there is only one copy
of the global \snippet{x} that both threads will be touching, there could be a problem caused by
those operations.

Local objects are not immune to problems since any function can start a new thread and pass a
pointer to a local object as a parameter to that thread. For example

\begin{lstlisting}
void f()
{
  std::string x;

  start_thread(some_function, &x);
  start_thread(some_function, &x);
}
\end{lstlisting}

Here I assume there is a library function named \snippet{start\_thread()} that accepts a pointer
to a thread function (defined elsewhere) and a pointer to a parameter for that function. In this
case I start two threads executing \snippet{some\_function()}, giving both of them a pointer to
the string \snippet{x}. If \snippet{some\_function()} tries to modify that string then two
threads will be modifying the same object and problems are likely. Note that this case is
particularly insidious because \snippet{some\_function()} has no particular reason to expect
that it will be given the same parameter twice. Thus it is unlikely to have any protection to
handle such a case.

\subsection{Writing Thread Safe Code}
\label{subsec:writing-safety}

In theory the only way to control the actions of a thread is to use synchronization primitives
such as mutexes or semaphores. In languages that provide threads as a part of the language,
synchronization primitives of some kind are normally provided by the language itself. In other
cases, such as with C, they are library functions, such as the POSIX API described in this
tutorial, that interact with the operating system.

Normally you should write code that meets the usual expectations that people have about thread
safe code. If you are implementing a C++ class, make sure that multiple simultaneous reads on an
object are safe. If you do update internal data behind the caller's back, you will probably have
to protect those updates yourself. Also make sure the simultaneous writes to independent objects
are safe. If you do make use of shared data, you will probably have to protect updates to that
shared data yourself. If you write a function that manages shared resources for multiple threads
from a common pool of such resources, you will probably have to protect the resource pool from
corruption by multiple, simultaneous requests. However, in general, you probably don't have to
bother protecting every single object against simultaneous updates. Let the calling program
worry about that case. Such total safety is usually very expensive in terms of runtime
efficiency and not normally necessary or even appropriate.

\subsection{Exception Safety vs Thread Safety}
\label{exception-safety}

Both thread and exception safety share a number of common issues. Both are concerned with
objects that are in an inconsistent state. Both have to think about resources (although in
different ways... exception safety is concerned with resource leaks, thread safety is concerned
with resource corruption). Both have several levels of safety that could be defined along with
some common expectations about what is and is not safe.

However, there is one important difference between the exception safety and thread safety.
Exceptions occur synchronously with the program's execution while threads are asynchronous. In
other words, exceptions occur, in theory, only at certain well defined times. Although it is not
always clear which operations might throw an exception and which might not, in theory it is
possible to precisely define exactly when an exception might happen and when it can't happen. As
a result it is often possible to make a function exception safe by just reorganizing it. In
contrast there is no way to control when two threads might clash. Reorganizing a function is
rarely helpful when it comes to making it thread safe. This difference makes thread related
errors difficult to reproduce and difficult to manage.


\section{Rules for Multithreaded Programming}
\label{sec:rules}

In this section I'll try to summarize a few ``rules of thumb'' that one should keep in mind when
building a multithreaded program. Although using multiple threads can provide elegant and
natural solutions to some programming problems, they can also introduce race conditions and
other subtle, difficult to debug problems. Many of these problems can be avoided by following a
few simple rules.

\subsection{Shared Data}

As I described in Section \ref{sec:thread-synchronization}, when two threads try to access the
same data object there is a potential for problems. Normally modifying an object requires
several steps. While those steps are being carried out the object is typically not in a well
formed state. If another thread tries to access the object during that time, it will likely get
corrupt information. The entire program might have undefined behavior afterwards. This must be
avoided.

\subsubsection{What data is shared?}

\begin{enumerate}

\item Static duration data (data that lasts as long as the program does). This includes global
  data and static local data. The case of static local data is only significant if two (or more)
  threads execute the function containing the static local at the same time.

\item Dynamically allocated data that has had its address put into a static variable. For
  example, if a function uses malloc() or new to allocate an object and then places the address
  of that object into a variable that is accessible by more than one thread, the dynamically
  allocated object will then be accessible by more than one thread.

\item The data members of a class object that has two (or more) of its member functions called
  by different threads at the same time.
   
\end{enumerate}

\subsubsection{What data is not shared?}

\begin{enumerate}

\item Local variables. Even if two threads call the same function they will have different
  copies of the local variables in that function. This is because the local variables are kept
  on the stack and every thread has its own stack.

\item Function parameters. In languages like C the parameters to functions are also put on the
  stack and thus every thread will have its own copy of those as well.

\end{enumerate}

Since local variables and function parameters can't be shared they are immune to race
conditions. Thus you should use local variables and function parameters whenever possible. Avoid
using global data. Be aware, however, that \emph{taking the address of a local variable and
  passing that address to place where another thread can read it amounts to sharing the local
  variable with that other thread}.

\subsubsection{What type of simultaneous access causes a problem?}

\begin{enumerate}

\item Whenever one thread tries to update an object, no other threads should be allowed to touch
  the object (for either reading or writing). Mutual exclusion should be enforced with some sort
  of mutex-like object (or by some other suitable means).
   
\end{enumerate}

\subsubsection{What type of simultaneous access is safe?}

\begin{enumerate}

\item If multiple threads only read the value of an object, there should be no problem. Be
  aware, however, that complicated objects often update internal information even when, from the
  outside, they are only being read. Some objects maintain a cache or keep track of usage
  statistics internally even for reads. Simultaneous reads on such an object might not be safe.

\item If one thread writes to one object while another thread touches a totally independent
  object, there should be no problem. Be aware, however, that many functions and objects do
  share some data internally. What appears to be two separate objects might really be using a
  shared data structure behind the scenes.

\item Certain types of objects are updated in an uninterruptable way. Thus simultaneous reads
  and writes to such objects are safe because it is impossible for the object to be in an
  inconsistent state during the update. Such updates are said to be \newterm{atomic}. The bad
  news is that the types that support atomic updates are usually very simple (for example: int)
  and there is no good way to know for sure exactly which types they are. The C standard
  provides the type \snippet{sig\_atomic\_t} for this purpose. It is defined in
  \snippet{<signal.h>} and is a kind of integer. Simultaneous updates to an object declared to
  be \snippet{volatile sig\_atomic\_t} are safe. Mutexes are not necessary in this case.

\end{enumerate}

\subsection{What can I count on?}

Unless a function is specifically documented as being thread-safe, you should assume that it is
not. Many libraries make extensive use of static data internally and unless those libraries were
designed with multiple threads in mind that static data is probably not being properly protected
with mutexes.

Similarly you should regard the member functions of a class as unsafe for multiple threads
unless it has been specifically documented to be otherwise. It is easy to see that there might
be problems if two threads try to manipulate the same object. However, even if two threads try
to manipulate different objects there could still be problems. For various reasons, many classes
use internal static data or try to share implementation details among objects that appear to be
distinct from the outside.

You can count on the following:

\begin{enumerate}

\item The API functions of the operating system are thread-safe.

\item The POSIX thread standard requires that most functions in the C standard library be
  thread-safe. There are a few exceptions which are documented as part of the standard.

\item Under Windows the C standard library is totally thread safe provided you use the correct
  version of the library and you initialize it properly.

\item The thread safety of the C++ standard library is vague and very much dependent on the
  compiler you are using. The SGI criteria for thread safety of the standard template library is
  gaining ground as the defacto standard. It is not universal.

\end{enumerate}

If you use a non thread-safe function in one of your functions, your function will be rendered
non thread-safe as well. However, you are free to use a non thread-safe function in a
multithreaded program provided it is never called by two or more threads at the same time. You
can either arrange to use such functions in only one thread or protect calls to such functions
with mutexes. Keep in mind that many functions share data internally with other functions. If
you try to protect calls to a non thread-safe function with a mutex you must also protect calls
to all the other related functions with the same mutex. This is often difficult.


\end{document}
